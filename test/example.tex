\documentclass{article}
\title{\textbf {Precautions and useful methods concerning managing differences in  cross-cultural teams}}
\date{}
\author{LIU Siqi}
\begin{document}
\maketitle

\section{Introduction}

Sometimes a team can be made up of people from different countries, which means that they may be growing up in different culture environments, causing different understanding to a task assigned. Just as what is expressed in the article written by Merce Mach et al, when people receive a brand-new task, they tend to try to understand it and perform the work according to what they have experienced in the past, thus bringing some effects to other members' performances (Mech Mach, 2015). \cite{mach2015team} As what is stated in the essay written by Ascalon, our world is developing at a high speed so it appears inevitable to gather an international team (Ascalon, 2008). \cite{evelina2008cross} Hence, some knowledge is supposed to be knew about what should be fulfilled to manage the problems may come across when faced with a cross cultural teams.

\section{Main Content}

Here comes the question about how to solve the problems people may have met in the process of building a cross cultural team and how to promote the corporation between the team members. Culture, which plays a role of an external source of impact influencing the team members may have the result of affecting how they act or behave when gathered together with each other, say, their willingness of corporate may be changed attributed to the ``outside world" they bring to the place where they work or study (Dong, 2010).\cite{dong2010cross}

\subsection{Trust}

It is convinced that all the team members have to trust each other. Trust, as a decisive factor in a team which wants to do some effective work, really matters when considering whether a team can express the best they can perform (Merch Mach ,2015). \cite{mach2015team} Throughout the process of team work, everyone cannot fulfill the tasks all by himself or herself, hence it is then inevitable to discover whether our partners can offer some help. Then, it becomes essential to put trust on others, believing in their ability to give us some suggestions or methods to complete the assignments on our hands. For people from different cultures, trust becomes especially important.

\subsection{Learning Initiatively}

People all round the globe are believed to be connected with each other closer and tighter. Hence, the significance of practising the skills of contacting with people around us who come from different cultures tends to be so important and crucial (Phillip W. Balsmeier, 1994). \cite{balsmeier1994cross}

\noindent According to which is stated above, to initiatively learn something about foreign cultures can give a great help of having a 
greater understanding of cross-culture and to distinguish the differences. Nothing would be improved if someone refuses to learn and discover on his own. For individuals, actively finding information and reading to learn about cultural differences is the basis for teamwork and collaboration, and is also fundamental to managing problems in cross-cultural teams.

\subsection{Communication}

In the paper written by Merce Mach and Yehuda Baruch, they argue that team members need to achieve a collective orientation to raise team?s performance and it is convinced to have the tendency to be a decisive factor which influences individuals to some extent. What?s more, what is described in this essay shows a fact to us that people might put their own interest away and tend to pay more attention to what concerning the whole team, given the hypothesis that the team consists of people with great sense of collectivism orientation (Merce Mach,2015). \cite{mach2015team}

\noindent To get collectivist oriented, communication cannot be neglected. People are collective animals. When forming a team, everyone needs to have a certain understanding and understanding of each other. At the same time, under the request of collectivism, it is even more important to communicate and communicate well. 

\noindent Communication within the team is indispensable because it allows everyone to understand the progress and perceptions of other members, so that they can adjust what they are doing and make the team more cohesive. Especially in a cross-cultural team, when members encounter certain problems, it is especially important to communicate effectively. It is normal for a particular problem encountered by a member not to be understood by other members from different cultural backgrounds. Therefore, at this time, effective communication and communication are needed to make the problem clear to every member. At this time, it is possible to further seek solutions.

\section{Conclusion}
The methods of managing differences in cross cultural teams discussed in my essay are just a very small part of all the creative or inspiring solutions. In short, it can be concluded that what counts the most is understanding and trust, together with learning initiatively and having effective communications. If those can be achieved in a cooperation, then the team work may become more efficient than before.

\bibliographystyle{plain}
\bibliography{bibfile}  
\end{document}